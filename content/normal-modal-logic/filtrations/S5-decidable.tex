% Part: normal-modal-logic
% Chapter: filtrations
% Section: S5-decidable

\documentclass[../../../include/open-logic-section]{subfiles}

\begin{document}

\olfileid{nml}{fil}{dec}

\olsection{\Log{S5} is Decidable}
The finite model property gives us an easy way to show that systems of
modal logic given by schemas are \emph{decidable} (i.e., that there is
a computable procedure to determine whether !!a{formula} is !!{derivable} in
the system or not).

\begin{thm}
  \Log{S5} is decidable.
\end{thm}

\begin{proof}
  Let $!A$ be given, and suppose the propositional variables
  occurring in $!A$ are among $p_1$, \dots, $p_k$.  Since for each
  $n$ there are only finitely many models with $n$ worlds assigning a
  value to $p_1$, \dots, $p_k$, we can enumerate, \emph{in parallel}, all
  the theorems of \Log{S5} by generating proofs in some systematic
  way; and all the models containing $1$, $2$, \dots worlds and checking
  whether $!A$ fails at a world in some such model. Eventually
  one of the two parallel processes will give an answer, as by
  \olref[com][fra]{thm:generaldet} and \olref[fmp]{cor:S5fmp}, either
  $!A$ is !!{derivable} or it fails in a finite universal model.
\end{proof}

The above proof works for \Log{S5} because filtrations of universal
models are automatically universal. The same holds for reflexivity and
seriality, but more work is needed for other properties.

\end{document}
